\begin{table}[h]
  \begin{tabular}{L{15cm} L{3.5cm}}
    \parttitle{Master Thesis} \\

    \textbf{Investigating the effects of discreteness in N-body simulations \
    of cosmological structure formation dominated by cold dark matter} & \multirow{5}{*}{ECD: Mar 2017} \\
    \desc{Supervisor: Prof. Dr. Cristiano Porciani} & \\

    \sep

    \desc{The fact that cosmological N-body simulations, instead of solving the
    Vlasov-Poisson equation, compute the gravitational interactions between
    discrete particles, leads to some unphysical consequences in the final
    results (e.g. spurious structures).}& \\
    \desc{Our approach to study the impacts of discreteness is to generate
    controlled initial conditions, evolve them using N-body simulations and
    observe and analyse the properties of the halos along the history of the
    simulation/Universe.} & \\
    \desc{One of our main goals is to answer the following question: Which
    criteria one should follow to get a desired precision in the results of a
    cosmological N-body simulation?} & \\
  \end{tabular}
\end{table}