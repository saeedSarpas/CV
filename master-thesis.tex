\begin{longtable}{L{15cm} L{3.5cm}}
  \parttitle{2}{Master Thesis} \\

  \textbf{Discreteness effects in N-body simulations of cosmological structure
    formation} & \multirow{5}{*}{ECD:~Mar 2017} \\
  \desc{\AIfA} & \\
  \desc{Supervisor: Prof.\ Dr.\ Cristiano Porciani} & \\

  \sep

  \desc{Cosmological N-body simulations compute the gravitational
    interactions between discrete particles instead of solving the Vlasov-Poisson
    equation. This discreteness might lead to some nonphysical consequences
    (e.g.~generation of spurious structures, noisy force calculation, regular
    fragmentation).}& \\
  \desc{Our approach to studying these effects is,} \\
  \initem{1.~generate different initial conditions (IC) by varying the number of
    particles at fixed power spectrum,}\\
  \initem{2.~run N-body simulations using generated ICs,}\\
  \initem{3.~find matching halos between different simulations,}\\
  \initem{4.~study the statistical properties of the halos along the history of
    the simulations.}\\
\end{longtable}