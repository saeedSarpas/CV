\begin{longtable}{L{15cm} L{3.5cm}}
  \parttitle{2}{Master Thesis} \\

  \textbf{Discreteness effects in N-body simulations of cosmological structure
    formation} & \multirow{5}{*}{ECD:~Mar 2017} \\
  \desc{\AIfA} & \\
  \desc{Supervisor: Prof.\ Dr.\ Cristiano Porciani} & \\

  \sep

  \desc{Cosmological N-body simulations solve the collisionless Boltzmann
    equation in an approximate way by sampling phase-space with discrete
    particles and computing the gravitational interactions between them. This
    (macro-)particles are many orders of magnitude more massive than any
    plausible dark-matter candidate and it is thus necessary to fine tune the
    force resolution in order to suppress few-body relaxation effects. All this
    might corrupt the output with unphysical components ranging from the
    contamination of physical quantities with Poisson noise to the creation of
    spurious halos. In this project, we quantify the importance of
    discreteness effects by finding and comparing ``matching-halos'' in a set of
    cosmological simulations with different mass and force resolutions but
    identical initial conditions. Our results show a tight relationship between
    the number of particles in a virialized halo and the accuracy of the most of
    its properties.}
\end{longtable}