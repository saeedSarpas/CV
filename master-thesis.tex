\begin{longtable}{L{15cm} L{3.5cm}}
  \parttitle{2}{Master Thesis} \\

  \textbf{Investigating the effects of discreteness in the N-body simulations \
  of cosmological structure formation dominated by cold dark matter}
  & \multirow{6}{*}{ECD:~Mar 2017} \\
  \desc{Argelander-Institut f\"{u}r Astronomie, University of Bonn, Germany} & \\
  \desc{Supervisor: Prof.\ Dr.\ Cristiano Porciani} & \\

  \sep

  \desc{The fact that cosmological N-body simulations, instead of solving the
  Vlasov-Poisson equation, compute the gravitational interactions between
  discrete particles, leads to some nonphysical consequences in the final
  results (e.g.~spurious structures).}& \\
  \desc{Our approach to study the impacts of discreteness is to generate
  controlled initial conditions, evolve them using N-body simulations and
  observe and analyze the properties of the halos along the history of the
  simulation/Universe.} & \\
  \desc{One of our main goals is to answer the following question: Which
  criteria one should follow to get a desired precision in the results of a
  cosmological N-body simulation?} & \\
\end{longtable}