\begin{longtable}{L{15cm} L{3.5cm}}
  \parttitle{2}{Professional Experience} \\

  \textbf{Internship} \desc{(\AIfA)}& \multirow{3}{*}{Nov 2015 --- Feb 2016} \\
  \desc{A C code to the matter power spectrum} & \\

  \sep

  \desc{A modular C code for calculating the matter power spectrum of
  a given field, extendable to calculate bi-spectrum.
  For developing this code, I used Test-Driven Development (TDD) which makes
  the programmer confident of the code accuracy by writing tests before
  implementation.} & \\

  \sep

  \textbf{Teaching Assistant} \desc{(\AIfA)}& \multirow{2}{*}{Apr --- Sep 2016} \\
  \desc{Organized tutorial classes and solved problems for graduate-level course
  Astrophysics of Galaxies} & \\

  \sep

  \textbf{Teaching Assistant} \desc{(\AIfA)}
  & \multirow{2}{*}{Oct 2015 --- Mar 2016} \\
  \desc{Organized tutorial classes, solved problems and evaluated tests for
  graduate-level course Cosmology} & \\

  \sep

  \textbf{Bachelor Thesis} \desc{(\SBU)}
  & \multirow{4}{*}{Apr 2013} \\
  \desc{Simulating star clusters with NBODY6++} & \\
  \desc{Supervisor: Assoc. Prof.\ Dr.\ Sadegh Movahed} & \\

  \sep

  \desc{In this project, in addition to running sample simulations using NBODY6++,
  we studied different numerical methods for solving linear differential
  equations suitable for direct N-body simulations of star-cluster-size systems.}
  & \\

  \sep

  \textbf{Poster Presentation} \desc{(Institute for Advanced Studies in Basic
  Sciences (IASBS), Zanjan, Iran)} & \multirow{3}{*}{Apr 2012} \\
  \desc{Simulation of gravitational N-body systems fully implemented on graphics
  processing units with CUDA C} & \\
  \desc{Presented in the 16th annual meeting on research in Astronomy} & \\
\end{longtable}
